% Japanese Abstract
\begin{jabstract}

DMAを用いたメモリフォレンジックの高いステルス性を活かしたマルウェアの検知,解析手法が研究されている. CPUが進化し多くのホスト上で複数のVMが動作することが増えているため,VMにこの手法を適用する利点は大きいが,既存手法ではこれは困難である.本研究ではこの問題をKVM・Linux環境において解決する.シンボルテーブルを圧縮して格納しているデータ構造をメモリ中から探索,展開することでvCPUを管理するデータ構造へのアクセスを実現し,EPTを復元することでVMのメモリ空間へのアクセスを実現する.カーネルのバージョン,コンフィグレーションを変化させた複数の環境を用意し,提案手法でVMのメモリ空間へのアクセスが可能になったことを評価した.

\end{jabstract}


% English Abstract
\begin{eabstract}

Malware detection and analysis methods that take advantage of the high stealth of memory forensics using DMA are being studied. As CPUs evolve and multiple VMs are increasingly running on many hosts, it is highly advantageous to apply this technique to VMs. However, this is difficult with existing methods. In this study, we solve this problem in a KVM/Linux environment. We access the data structure that manages the vCPU by searching and extracting the data structure that stores the compressed symbol table from memory, and access the memory space of the VM by restoring the EPT. We evaluated that the proposed method can access the memory space of VMs by preparing multiple environments with different kernel versions and configurations.

\end{eabstract}
