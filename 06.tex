\chapter{結論}
\label{chap:conclustion}


本研究ではIntelプロセッサ上のKVM・Linux環境において
DMAによるメモリフォレンジックをVMに適用するために,
KVM,Linuxのソースコードを解析.......
本研究の成果により,従来は困難であった
DMAによるステルス性の高いマルウェア検知,解析手法を
KVM仮想マシンに適用することが可能となった.

Rutkowska,JoannnaはDMAによるメモリ取得を阻止する手法を提案している\cite{beyond_the_cpu}.
しかし,このような攻撃を実現する難易度は極めて高いことに対し,多くのマルウェアは
仮想マシン上での動作を検知する機能を既に実装している\cite{linux_malware}.
また,IOMMUが適切に設定された環境においては提案手法を使用することが出来ない.
しかし,殆どのLinuxディストリビューションにおいてはIOMMUはデフォルトでは有効化されていない.
Windows10においても同様であり\cite{win10iommu},Windows11においてもEnterprise,Serverエディションに
おいて設定をすれば使用できるといった状態に留まっている.デフォルトでIOMMUを有効化している
OSはApple macOS High Sierraのみである.
そのため,IOMMUの存在が一概に本研究の貢献を否定するものではないと考える.

今後は提案する手法が一般的なものであることを更に証明するため
評価環境を増やして更に実験を行う予定である.


