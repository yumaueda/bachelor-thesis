% 独自のコマンド

% ■ アブストラクト
%  \begin{jabstract} 〜 \end{jabstract}  :日本語のアブストラクト
%  \begin{eabstract} 〜 \end{eabstract}  :英語のアブストラクト

% ■ 謝辞
%  \begin{acknowledgement} 〜 \end{acknowledgement}

% ■ 文献リスト
%  \begin{bib}[100] 〜 \end{bib}


\newif\ifjapanese

\japanesetrue  % 論文全体を日本語で書く(英語で書くならコメントアウト)

\ifjapanese
  \documentclass[a4j,twoside,openright,11pt]{jreport} % 両面印刷の場合。余白を綴じ側に作って右起こし。
  %\documentclass[a4j,11pt]{jreport}                  % 片面印刷の場合。
  \renewcommand{\bibname}{参考文献}
  \newcommand{\acknowledgementname}{謝辞}
\else
  \documentclass[a4paper,11pt]{report}
  \newcommand{\acknowledgementname}{Acknowledgment}
\fi
\usepackage{thesis}
\usepackage{ascmac}
\usepackage[dvipdfmx]{graphicx}
\usepackage{multirow}
\usepackage{url}
\bibliographystyle{jplain}

\bindermode  % バインダー用余白設定

% 日本語情報(必要なら)
\jclass  {卒業論文}                             % 論文種別
\jtitle    {KVM 仮想マシンへの DMA によるメモリフォレンジック手法の提案と実装}    % タイトル。改行する場合は\\を入れる
\juniv    {慶應義塾大学}                  % 大学名
\jfaculty  {総合政策学部}               % 学部,学科
\jauthor  {上田 侑真}                       % 著者
\jhyear  {4}                                   % 年度(令和)
\jsyear  {2022}                                 % 年度(西暦)
\jkeyword  {メモリフォレンジック,DMA,仮想マシン,KVM}     % 論文のキーワード
\jproject{村井・中村・楠本・高汐・バンミーター・植原\\三次・中澤・手塚・武田・大越 合同研究プロジェクト} %プロジェクト名
\jdate{2022年7月}

% 英語情報(必要なら)
\eclass  {Graduation Thesis}                            % 論文種別
\etitle    {Enabling DMA-based Memory Forensics to Virtual Machines on KVM}      % タイトル。改行する場合は\\を入れる
\euniv  {Keio University}                             % 大学名
\efaculty  {Faculty of Policy Management}  % 学部,学科
\eauthor  {Yuma Ueda}                           % 著者
\eyear  {2022}                                        % 西暦○年度
\ekeyword  {Memory Forensics, DMA, Virtual Machine, KVM}          % 論文のキーワード
\eproject{Murai / Nakamura / Kusumoto / Takashio / Van Meter / Uehara / Mitsugi / Nakazawa / Tezuka / Takeda / Okoshi Joint Research Group}                 %プロジェクト名
\edate{July 2022}





\begin{document}

\ifjapanese
  \jmaketitle    % 表紙(日本語)
\else
  \emaketitle    % 表紙(英語)
\fi

% Japanese Abstract
\begin{jabstract}

DMAを用いたメモリフォレンジックの高いステルス性を活かしたマルウェアの検知,解析手法が研究されている. CPUが進化し多くのホスト上で複数のVMが動作することが増えているため,VMにこの手法を適用する利点は大きいが,既存手法ではこれは困難である.本研究ではこの問題をKVM・Linux環境において解決する.シンボルテーブルを圧縮して格納しているデータ構造をメモリ中から探索,展開することでvCPUを管理するデータ構造へのアクセスを実現し,EPTを復元することでVMのメモリ空間へのアクセスを実現する.カーネルのバージョン,コンフィグレーションを変化させた複数の環境を用意し,提案手法でVMのメモリ空間へのアクセスが可能になったことを評価した.

\end{jabstract}


% English Abstract
\begin{eabstract}

Malware detection and analysis methods that take advantage of the high stealth of memory forensics using DMA are being studied. As CPUs evolve and multiple VMs are increasingly running on many hosts, it is highly advantageous to apply this technique to VMs. However, this is difficult with existing methods. In this study, we solve this problem in a KVM/Linux environment. We access the data structure that manages the vCPU by searching and extracting the data structure that stores the compressed symbol table from memory, and access the memory space of the VM by restoring the EPT. We evaluated that the proposed method can access the memory space of VMs by preparing multiple environments with different kernel versions and configurations.

\end{eabstract}
  % アブストラクト。要独自コマンド,include先参照のこと

\tableofcontents  % 目次
\listoffigures    % 表目次
\listoftables    % 図目次

\pagenumbering{arabic}

\chapter{序論}
\label{chap:introduction}

\section{背景}
\label{sec:background}

新たなマルウェアの検知,解析手法の開発に伴い
マルウェアはより高度な検知,解析回避手法を搭載するようになっている.
例えばルートキットはOSの重要な構造体を改変する
DKOM(Direct Kernel Object Manipulation)\cite{dkom}と呼ばれる手法で
自身の存在を隠蔽することがある\cite{florio}.
このようなマルウェアに感染した場合
ホストOSを経由して得た情報は信用することが出来ない.
そのため,セキュリティ研究者はメモリ上のデータそのものを取得し,解析することで
マルウェアの検知や解析を行うことを試みてきた.
このような手法の多くは
カーネルモジュールや
VMMからゲストOSの状態を監視するVirtula Machine Introspection(VMI)
を用いてメモリ取得を行うものであった\cite{lime} \cite{vmwatcher}.
しかし,このような手法で取得したメモリ上のデータは不正なものである可能性がある.
Sparks,BatlerはWindowsのアドレス変換プロセスとTLBの不整合を悪用することで
メモリ上の悪意のあるコードを隠蔽する手法を開発した\cite{shadowwalker}.
Palutke, Freilingはこの手法を拡張しIntel VT-xが提供するEPTを利用して
メモリ上のデータを隠蔽する手法を提案した\cite{styx}.
これらの手法はホスト上のOS,CPUを経由して取得したメモリ上のデータが
改竄されている可能性を考慮しなければならないことを示している.
また,マルウェアは自身が仮想マシン上で動作していることを検知すると
悪性な動作を停止する場合がある\cite{linuxmalware}.

このような背景からマルウェアの検知,解析においては,それらを行うシステムは
マルウェアが存在しうるレイヤ上で動作するべきでなく,マルウェアに存在を察知
されてはならないことがわかる.

\section{問題}

第\ref{sec:background}で述べたような背景から,
高いステルス性を持つDMAによるメモリ取得を用いて
ホスト上のマルウェアを検知,解析し,更には対処を行うといった研究が
行われてきた\cite{firewire}\cite{hardware_based_memory_acquistion}\cite{amoeba}\cite{anti_forensic_resilient_memory_acquisition}\cite{lo-phi}.
CPUのマルチコア化が進み単一のマシン上に複数のVMが動作することが増えているため
これらの手法をVMに適用する利点は大きい一方,物理メモリ中のデータから仮想マシンの
構成やVMのメモリ空間を復元する有効な手法が知られていないため,現状これは困難である.
本研究はこの問題をIntelプロセッサ上のKVM・Linux環境で解決する手法を提案し,その有効性を
評価するものである.

\section{貢献}

本研究の貢献は以下の通りである.

\begin{enumerate}

    \item Intelプロセッサ上のKVM・Linux環境において物理メモリ中のデータからVMのメモリ空間を復元する手法を提案し,有効性を確認した.
    \item その結果,従来は困難であったDMAによるステルス性の高いマルウェア検知,解析手法をKVM仮想マシンに適用することが可能となった.

\end{enumerate}


\section{本稿の構成}

本稿の構成は以下の通りである.第\ref{chap:related_works}章では関連研究・技術について
概説した後に既存のメモリフォレンジックをVMに適用する手法や,その問題点について
述べる.第\ref{chap:approach}章ではKVM上のVMにメモリフォレンジックを適用する
ための提案手法の概観について述べ,第\ref{chap:implementation}章ではその詳細な
実装について解説する.第\ref{chap:evaluation}章で提案手法の有効性を評価し,
第\ref{chap:conclusion}章にて実験結果についての議論,今後の展望を含む
本研究の総括を行う.

  % 序論
\chapter{関連研究・関連技術}
\label{chap:related_works}

本章では,本研究の提案手法を述べる前に関連研究や関連技術,その問題点について記述する.

\section{Intel VT: Virtualization Technology}
\label{sec:vt}

Intelは2005年にx86アーキテクチャ上での仮想マシンの実行をサポートし,仮想化オーバーヘッドを
削減するための技術である
VTを発表した\cite{intelsdm}.VTはハードウェアレベルで仮想マシンの実行をサポートするいくつかの
コンポーネントからなり,そのうちの1つにVT-xがある.

VT-xはCPUやメモリの仮想化を支援する機構であり,後述するVMCSや
EPTはVT-xの機能である.
VT-xでは仮想化を支援するためにプロセッサにVMX root,VMX non-rootの2つの動作モードが
設定された.
VMX rootモードはVMMが動作するモードであり,VMX non-rootモードはゲストOSが動作するモードである.
ゲストOSがハードウェアリソースへのアクセスなどのセンシティブ命令を実行した場合,VMEXITと
呼ばれるVMX rootモードへの遷移が発生し,VMMが必要な処理を行うことが出来る.
VMX non-rootモードへの遷移はVMENTRYと呼ばれる.

\section{VMCS: Virtual-Machine Control Structure}
\label{sec:vmcs}

CPUはVMCSと呼ばれる構造体を用いてVMX root,VMX non-rootモードそれぞれへの遷移や
VMX non-rootモードでの動作時の挙動を管理している\cite{intelsdm}.
VMCSはホストのメモリ上に確保され,VMPTRLD命令によってCPUに登録される.
VMWREAD,VMWRITEといったVMX命令を用いてVMCSの各フィールドの値を読み書きすることができる.

\section{EPT: Extended Page Table}
\label{sec:ept}

Intel CPUはNehalemアーキテクチャ以降,ゲスト物理アドレス(GPA)からホスト物理アドレス(HPA)への
アドレス変換をハードウェアで支援するEPTを採用している\cite{nehalem}.
vCPUが物理メモリ上のデータにアクセスする場合
GPAを利用してメモリアクセスを行おうとする.EPTを利用する環境においては,CPUがEPTを参照し,
GPAをHPAに置き換えてメモリアクセスを実施する.
これによりシャドウページテーブルを用いた既存手法と比較して,GPAからHPAへのアドレス変換が大幅に高速化された.
VMWRITE命令を用いてVMCSにEPTのポインタを登録することで,CPUはEPTのメモリ上での位置を知り,利用することが出来る.

\section{ネスト仮想化}
\label{sec:nested_virtualization}

ネスト仮想化はPopek, Goldberg\cite{popek_goldberg}によって提唱された,
仮想マシン上で仮想マシンを実行するという概念である.
しかし,x86アーキテクチャではネスト仮想化がサポートされないため,どれだけネストされた
仮想マシンであっても1つのVMMがトラップをハンドリングすることになる.
そのため,ネスト仮想化に関する処理はVMM中に実装される必要がある.
KVMにおけるネスト仮想化はBen-Yehudaらによって実装された\cite{turtles}.

(まだ詳細を書き足す)
(Nested-VMのvCPUに対応するVMCSがベアメタルマシン上で動作するホストOS上に存在することを記述する
必要がある)

\section{既存手法}
\label{subsec:existing_method}

VMにメモリフォレンジックを行うためにはVMのメモリ空間を復元しなければならない.
これを実現するためにはVMやそのvCPUの数,EPTのアドレス,ネスト仮想化の有無
といった情報を把握する必要がある.Intel VT環境においてはこのような情報は
VMCSから得ることができる.

しかし,VMCS内の各フィールドのメモリ上でのオフセットはアーキテクチャによって異なる上,
公開されていない.
フィールドへの読み書きはVMREAD,VMWRITE命令を通じて行われるため
VMMがこれを意識することはないが,
この仕様はシグネチャマッチングなどを用いたVMCSのメモリ中からの発見を
困難にしている.

Grazianoら\cite{hypervisor_memory_forensics}はVMCS領域に
値をインクリメントした2バイト単位のデータ列を配置した後,
VMREAD命令を用いて各フィールドの値を読み出すことで
VMCSのメモリレイアウトを復元することに成功した.これによりVMCSをメモリ中から
探索することが可能になった.復元されたVMCSに格納されている
物理CPU,vCPUのCR3レジスタの値を比較することでVMの構成を復元,
EPTPフィールドを読み出すことでvCPUが使用するEPTのアドレスを取得し,
VMにメモリフォレンジックを行うことが可能になった.
Zhangら\cite{kvm_vm_forensics}はこの手法を用いてKVM上の仮想マシンの
詳細な情報を取得することに成功した.

\section{既存手法の問題}
\label{subsec:problem_with_existing_method}

IntelプロセッサはCPUの内部にVMCSを格納するための
記憶領域を保持している.そのため,VMWRITE命令を用いて
VMCSSのフィールドに値を書き込んだとしても,メモリ上のVMCS領域に値が反映される
保証はない\cite{intelsdm}\cite{rekall}(図\ref{fig:problem}).Grazianoらが提案した手法
\cite{hypervisor_memory_forensics}やそれを利用した後続の研究\cite{kvm_vm_forensics}では
VMCSをシグネチャマッチングでメモリ上から探索しているため,
このような機構が存在するXeon WestmereやHaswell以降の
比較的モダンなIntelプロセッサには適用できないという問題がある.

\begin{figure}[h]
  \includegraphics[scale=0.305]{problem.png}
  \caption{CPU内に存在するVMCS記憶領域}
  \label{fig:problem}
\end{figure}
  % 背景
\chapter{提案手法の概要}
\label{chap:system_overview}

i
本章では,まず提案手法を実現するためのシステムの概要を示し,
次にシステムを構成する各コンポーネントの詳細について述べる.

本研究で提案するシステムはMonitor Host,Target Hostの2ホスト間で動作し,
NetTLP\cite{nettlp},Kallsyms Extractor,VM Memory Mapperの3つのコンポーネントからなる.

\begin{figure}[h]
  \includegraphics[scale=0.305]{system_overview.png}
  \caption{提案手法の概要}
  \label{fig:sysytem_overview}
\end{figure}

Target HostはIntelプロセッサを搭載し,ホストOSはLinuxであり,KVM仮想マシンが実行されている.
NetTLPは提案するシステムの中で,Target Hostの物理メモリ空間を取得しMonitor Hostに
転送する役割を担う.
Kallsyms ExtractorはNetTLPを用いて取得したTarget Hostの物理メモリ上のデータから
ホストOSのシンボルテーブルを復元する.
VM Memory Mapperは復元されたホストOSのシンボルテーブルを用いて
ホストOS上の各VMを管理する構造体にアクセスし,EPTを復元する.その後,復元したEPTを走査し
内部にGPAからHPAのマップを保持する.
Applicationは本システムを利用し,VMへDMAによるメモリフォレンジックを行うアプリケーションである.
ApplicationがあるGPA上のデータの取得を要求する際,VM Memory Mapperを用いて
GPAをHPAに変換することでVMのメモリ空間へのアクセスを実現している.


\section{NetTLP}

システムのうちTarget Hostのメモリ取得を担う部分としてNetTLP\cite{nettlp}を利用した.
NetTLPは本来PCIeデバイスのプロトタイピングを想定したプラットフォームである.
NetTLP Adapterと呼ばれるPCIeデバイスをインストールしたホストのルートコンプレックスと
NetTLP AdapterにEthernetで接続された外部のホストとの間で
UDPでカプセリングされたDMA RequestなどのTLPをやり取りすることが可能である.
また,TLPをカプセリングして送受信する処理を担うLibTLPも用意されている.
本システムではこれらを利用してTargetのルートコンプレックスにDMA Read Requestを送信し,
Target Hostの物理メモリ上のデータを取得している.

\section{Kallsyms Extractor}


Kallsyms ExtractorはNetTLPを経由して取得したTarget Hostの物理メモリ上のデータから
ホストOSのシンボルテーブルを復元する.

物理メモリ上のデータを解析するためにはカーネル内の構造体のアドレスを知る必要がある.
シンボルテーブルには多くの重要な構造体のアドレスが格納されており,これを復元するための
研究は多く行われてきた\cite{ksfinder}\cite{volatility_android}
\cite{zhang2017research}\cite{autoprofile}.しかし,これらのアプローチはシンボルテーブルの一部しか復元できないか,全てを復元できるがエミュレータを用いなるなど複雑な処理を実行する必要があるものであった.

そこで,Linuxカーネル内におけるシンボルテーブルの構造を利用し,
非常にシンプルな処理で全てのシンボルテーブルを復元することができるKallsyms Extractorを提案する.

以下にLinuxカーネル内におけるシンボルテーブルの構造を図示する(図\ref{fig:kallsyms}).
Linuxのシンボルテーブルは巨大であるため
シンボル名は圧縮されて格納される.シンボルのアドレスは8バイトのベースアドレスに対する
4バイトの相対アドレスとして表現され,データ量の削減が試みられている.
シンボル名はその一部の文字列と対応する1バイトのトークンに変換され,トークン数+トークン配列の形式でu8型の配列であるkallsyms\_namesに格納されている.
kallsyms\_namesに含まれるトークンをインデックスとしてu16型の配列kallsyms\_token\_indexに
アクセスすることでchar型の配列であるkallsyms\_token\_table内のインデックスを得られる.
kallsyms\_token\_table内のこのインデックスからnullが現れるまでがトークンに対応する文字列となる.
例えばkallsyms\_names[0]=2であればkallsyms\_names[1,3)に
シンボル名の一部に対応するトークンが格納されている.
kallsyms\_names[1,3)=\{0x00,0x01\}であれば1つめのシンボル名はhowmmとなる.
シンボルのアドレスはkallsyms\_relative\_baseからの相対アドレスとして配列kallsyms\_offsets
に格納されている.

kallsyms\_names,kallsyms\_token\_tableの値はカーネルのバージョンとコンフィグレーションが
一定であれば一意に定まる.
Kallsyms Extractorは予め調査したカーネルのバージョン,コンフィグレーションに対する
これらの配列の値のデータをシグネチャとして,メモリ上からシンボルテーブルを復元するために
必要なデータ構造である

\begin{enumerate}
  \item kallsyms\_names
  \item kallsyms\_token\_table
  \item kallsyms\_token\_index
  \item kallsyms\_num\_syms
  \item kallsyms\_relative\_base
  \item kallsyms\_offsets
\end{enumerate}

の位置を特定する.昨今のPCの物理メモリ空間は巨大であるが
探索する必要のある範囲はKASLRのオフセットを考慮した512MiBのカーネルテキスト領域のみであることと
対象とするカーネルについて少しでも事前情報があればシグネチャの候補を削減できることから
処理は軽量である.
これらのデータ構造のメモリ上での位置はビルドに使用されたコンパイラ・リンカ等によって
微妙に異なるが,そのパターンはごく僅かであるため使用したシグネチャに対するカーネルバージョン,コンフィグレーションに
対応するシンボルが復元されるまで総当りを行うことで問題なくシンボルテーブルの復元が行える.


\begin{figure}[h]
  \includegraphics[scale=0.45]{kallsyms.png}
  \caption{Linuxカーネル内におけるシンボルテーブルの構造(値は例)}
  \label{fig:kallsyms}
\end{figure}

\section{VM Memory Mapper}

VM Memory MapperはホストOSの物理メモリ上のデータからホストOS上で動作するVMの
メモリ空間を復元する.これはホストOS上のVM数,
各VMのvCPU数,使用しているEPTのアドレス,ネストの有無を求めることで実現できる.

以下VM Memory Mappperが上述の情報を求める手法を記述する(図\ref{fig:ept_pointer}).
VM Memory MapperはまずKallsyms Extractorが復元した
シンボルテーブルを用いてカーネルシンボルlinux\_bannerにアクセスし,カーネルのバージョンを
特定する.カーネルのバージョン情報を得たことによりKVMがVM,vCPUの管理に用いる
データ構造の定義を確定させることが出来る.
次にシンボルテーブルからカーネルシンボルvm\_listのアドレスを求める.
vm\_listはKVMがVMを管理するために各VMにつき1つ作成するkvm構造体の
LIST\_HEADマクロによるリンクドリストのアドレスを示している.
kvm構造体のリンクドリストの要素数から,Target Host上に存在する仮想マシンの数を
求めることが出来る.
kvm構造体は自身に対応するVMの各vCPUを表すkvm\_vcpu構造体のポインタの配列を
vcpusメンバに格納している.この時点でVMの有するvCPU数も判明する.
vCPUに関する情報のうちCPUアーキテクチャに固有のものは
kvm\_vcpu構造体の配下のkvm\_vcpu\_arch構造体であるarchメンバに格納される.
そしてkvm\_vcpu\_arch構造体のメンバであるkvm\_mmu構造体のroot\_hpaメンバから
各vCPUが使用するEPTのアドレスを求めることが出来る.
VM Memory MapperはEPTに対してPage Walkingを行い,GPAからHPAへのマッピングを
記憶し,ユーザにGPAからHPAへの変換を求められた場合にこれを使用する.

この手法においてアクセスされるカーネル内のデータ構造の定義中には
カーネルコンフィグレーションが影響する部分は非常に少なく,影響する場合も
本研究が対象とするIntelプロセッサ上のKVM・Linux環境の範囲では
コンフィグレーションは一定である.
そのため各データ構造のメンバのメモリ中でのオフセットは
カーネルバージョン,ビルド環境につき一意に定まり
VM Memory Mapperはビルド環境による差異について総当り方式で試行を行うのみで
VMのメモリ空間を復元できる.


\begin{figure}[h]
  \includegraphics[scale=0.45]{ept_pointer.png}
  \caption{EPTへのポインタの格納位置}
  \label{fig:ept_pointer}
\end{figure}
  % 提案手法
\chapter{実装}
\label{chap:implementation}

本章では第\ref{chap:system_overview}章で述べた提案手法を
実現するシステムの詳細な実装について述べる.

\section{Kallsyms Extractor}

Kallsyms Extractorは第\ref{chap:system_overview}章で述べた手法を
実行するプログラムであり,C言語で実装されている.

Kallsyms Extractorの機能はシンボルテーブルの復元のみであり,
ユーザは復元されたシンボルテーブルを読み込むことさえできればよい.
よってシステムをシンプルなものにするため,Kallsyms Extractorは
復元したTarget Hostのシンボルテーブルをファイルとして出力する形式で
実装した.

\section{VM Memory Mapper}

VM Memory MapperはNetTLPを用いてTarget Hostのメモリから
VMのメモリ空間を復元するのみ
  % 実装
\chapter{評価}
\label{chap:evaluation}

本章では提案手法によりIntelプロセッサ上のKVM・Linux環境において
DMAによるメモリフォレンジックをVMに適用することが可能になったことを評価する.
評価に用いたTarget Hostのスペックは以下の通りである.
\begin{itemize}
  \item CPU: Intel Core i7-9700K @ 3.60GHz
  \item RAM: 32GiB
  \item OS: Ubuntu 18.04.6 LTS
  \item SSD: 250GiB
\end{itemize}
提案手法が一般的に適用できることを確認するため,
Target Hostのカーネルのバージョン,コンフィグレーションを変化させた
複数のKVM・Linux環境を用意した(表\ref{tab:evaluation_environment}).
更に各評価環境においてVMの構成を変化させながら提案手法を使用し,
VMのメモリ空間を復元する実験を行った.
評価に用いたVMの構成を以下に示す(表\ref{tab:vm_configuration}).なおVMのOSや
カーネルバージョン,コンフィグレーションは全てホストOSのものと同一である.
VMのメモリ空間が復元できたことは,VMのシンボルテーブルからlinux\_bannerの
GPAを求めた後に提案手法を用いて該当GPAにアクセスし.正しいバナーが表示される
ことをもって確認した.

実験の結果,全ての評価環境,全てのVMの構成において
VMのメモリ空間を復元できることが確認できた.

\begin{table}[htbp]
\caption{評価環境の一覧}
\label{tab:evaluation_environment}
\hbox to\hsize{\hfil
\begin{tabular}{l|ll}\hline\hline
& Kernel Version & Flavours \\\hline
環境1 & 4.15.0 & generic    \\
環境1 & 4.15.0 & lowlatency \\
環境2 & 4.18.0 & generic    \\
環境2 & 4.18.0 & lowlatency \\
環境3 & 5.0.0  & generic    \\
環境3 & 5.0.0  & lowlatency \\
環境3 & 5.4.0  & generic    \\
環境3 & 5.4.0  & lowlatency \\\hline
\end{tabular}\hfil}
\end{table}



\begin{table*}[htbp]
\caption{各評価環境におけるVMの構成一覧}
\label{tab:vm_configuration}
\hbox to\hsize{\hfil
\begin{tabular}{l|llll}\hline\hline
                & L1 VM数 & L1 VM RAM & L2 VM数              & L2 VM RAM \\\hline
Single          & 1       & 8GiB      & 0                    & N/A       \\
Multiple        & 2       & 8GiB      & 0                    & N/A       \\
Nested Single   & 1       & 8GiB      & 1                    & 4GiB      \\
Nested Multiple & 2       & 8GiB      & 2(1 on each L1 VM) & 4GiB each \\\hline
\end{tabular}\hfil}
\end{table*}
  % 評価
\chapter{結論}
\label{chap:conclustion}


本研究ではIntelプロセッサ上のKVM・Linux環境において
DMAによるメモリフォレンジックをVMに適用するために,
KVM,Linuxのソースコードを解析.......
本研究の成果により,従来は困難であった
DMAによるステルス性の高いマルウェア検知,解析手法を
KVM仮想マシンに適用することが可能となった.

Rutkowska,JoannnaはDMAによるメモリ取得を阻止する手法を提案している\cite{beyond_the_cpu}.
しかし,このような攻撃を実現する難易度は極めて高いことに対し,多くのマルウェアは
仮想マシン上での動作を検知する機能を既に実装している\cite{linux_malware}.
また,IOMMUが適切に設定された環境においては提案手法を使用することが出来ない.
しかし,殆どのLinuxディストリビューションにおいてはIOMMUはデフォルトでは有効化されていない.
Windows10においても同様であり\cite{win10iommu},Windows11においてもEnterprise,Serverエディションに
おいて設定をすれば使用できるといった状態に留まっている.デフォルトでIOMMUを有効化している
OSはApple macOS High Sierraのみである.
そのため,IOMMUの存在が一概に本研究の貢献を否定するものではないと考える.

今後は提案する手法が一般的なものであることを更に証明するため
評価環境を増やして更に実験を行う予定である.


  % 結論

\begin{acknowledgement}

本稿の執筆に際し,ご指導賜りました
慶應義塾大学名誉教授 村井純博士,
環境情報学部教授 中村修博士,
同学部教授 楠本博之博士,
同学部教授 高汐一紀博士,
同学部教授 Rodney D.Van Meter博士,
同学部教授 植原啓介博士,
同学部教授 三次仁博士,
同学部教授 中澤仁博士,
同学部教授 中澤仁博士,
同学部教授 手塚悟博士,
同学部教授 武田圭史博士,
同学部准教授 大越匡博士,
政策・メディア研究科特任教授 鈴木茂哉博士,
同研究科特任准教授 佐藤雅明博士
同研究科特任准教授 大江将史博士
同研究科特任助教 工藤紀篤博士,
同研究科特任講師 松谷健史博士,
同研究科特任講師 空閑洋平博士
に感謝申し上げます.

\end{acknowledgement}

\include{91_bibliography}
\appendix
%\include{92_appendix}    % 付録

\end{document}
