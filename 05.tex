\chapter{評価}
\label{chap:evaluation}

本章では提案手法によりIntelプロセッサ上のKVM・Linux環境において
DMAによるメモリフォレンジックをVMに適用することが可能になったことを評価する.
評価に用いたTarget Hostのスペックは以下の通りである.
\begin{itemize}
  \item CPU: Intel Core i7-9700K @ 3.60GHz
  \item RAM: 32GiB
  \item OS: Ubuntu 18.04.6 LTS
  \item SSD: 250GiB
\end{itemize}
提案手法が一般的に適用できることを確認するため,
Target Hostのカーネルのバージョン,コンフィグレーションを変化させた
複数のKVM・Linux環境を用意した(表\ref{tab:evaluation_environment}).
更に各評価環境においてVMの構成を変化させながら提案手法を使用し,
VMのメモリ空間を復元する実験を行った.
評価に用いたVMの構成を以下に示す(表\ref{tab:vm_configuration}).なおVMのOSや
カーネルバージョン,コンフィグレーションは全てホストOSのものと同一である.
VMのメモリ空間が復元できたことは,VMのシンボルテーブルからlinux\_bannerの
GPAを求めた後に提案手法を用いて該当GPAにアクセスし.正しいバナーが表示される
ことをもって確認した.

実験の結果,全ての評価環境,全てのVMの構成において
VMのメモリ空間を復元できることが確認できた.

\begin{table}[htbp]
\caption{評価環境の一覧}
\label{tab:evaluation_environment}
\hbox to\hsize{\hfil
\begin{tabular}{l|ll}\hline\hline
& Kernel Version & Flavours \\\hline
環境1 & 4.15.0 & generic    \\
環境1 & 4.15.0 & lowlatency \\
環境2 & 4.18.0 & generic    \\
環境2 & 4.18.0 & lowlatency \\
環境3 & 5.0.0  & generic    \\
環境3 & 5.0.0  & lowlatency \\
環境3 & 5.4.0  & generic    \\
環境3 & 5.4.0  & lowlatency \\\hline
\end{tabular}\hfil}
\end{table}



\begin{table*}[htbp]
\caption{各評価環境におけるVMの構成一覧}
\label{tab:vm_configuration}
\hbox to\hsize{\hfil
\begin{tabular}{l|llll}\hline\hline
                & L1 VM数 & L1 VM RAM & L2 VM数              & L2 VM RAM \\\hline
Single          & 1       & 8GiB      & 0                    & N/A       \\
Multiple        & 2       & 8GiB      & 0                    & N/A       \\
Nested Single   & 1       & 8GiB      & 1                    & 4GiB      \\
Nested Multiple & 2       & 8GiB      & 2(1 on each L1 VM) & 4GiB each \\\hline
\end{tabular}\hfil}
\end{table*}
