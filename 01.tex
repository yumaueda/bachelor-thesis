\chapter{序論}
\label{chap:introduction}

\section{背景}
\label{sec:background}

新たなマルウェアの検知,解析手法の開発に伴い
マルウェアはより高度な検知,解析回避手法を搭載するようになっている.
例えばルートキットはOSの重要な構造体を改変する
DKOM(Direct Kernel Object Manipulation)\cite{dkom}と呼ばれる手法で
自身の存在を隠蔽することがある\cite{florio}.
このようなマルウェアに感染した場合
ホストOSを経由して得た情報は信用することが出来ない.
そのため,セキュリティ研究者はメモリ上のデータそのものを取得し,解析することで
マルウェアの検知や解析を行うことを試みてきた.
このような手法の多くは
カーネルモジュールや
VMMからゲストOSの状態を監視するVirtula Machine Introspection(VMI)
を用いてメモリ取得を行うものであった\cite{lime} \cite{vmwatcher}.
しかし,このような手法で取得したメモリ上のデータは不正なものである可能性がある.
Sparks,BatlerはWindowsのアドレス変換プロセスとTLBの不整合を悪用することで
メモリ上の悪意のあるコードを隠蔽する手法を開発した\cite{shadowwalker}.
Palutke, Freilingはこの手法を拡張しIntel VT-xが提供するEPTを利用して
メモリ上のデータを隠蔽する手法を提案した\cite{styx}.
これらの手法はホスト上のOS,CPUを経由して取得したメモリ上のデータが
改竄されている可能性を考慮しなければならないことを示している.
また,マルウェアは自身が仮想マシン上で動作していることを検知すると
悪性な動作を停止する場合がある\cite{linuxmalware}.

このような背景からマルウェアの検知,解析においては,それらを行うシステムは
マルウェアが存在しうるレイヤ上で動作するべきでなく,マルウェアに存在を察知
されてはならないことがわかる.

\section{問題}

第\ref{sec:background}で述べたような背景から,
高いステルス性を持つDMAによるメモリ取得を用いて
ホスト上のマルウェアを検知,解析し,更には対処を行うといった研究が
行われてきた\cite{firewire}\cite{hardware_based_memory_acquistion}\cite{amoeba}\cite{anti_forensic_resilient_memory_acquisition}\cite{lo-phi}.
CPUのマルチコア化が進み単一のマシン上に複数のVMが動作することが増えているため
これらの手法をVMに適用する利点は大きい一方,物理メモリ中のデータから仮想マシンの
構成やVMのメモリ空間を復元する有効な手法が知られていないため,現状これは困難である.
本研究はこの問題をIntelプロセッサ上のKVM・Linux環境で解決する手法を提案し,その有効性を
評価するものである.

\section{貢献}

本研究の貢献は以下の通りである.

\begin{enumerate}

    \item Intelプロセッサ上のKVM・Linux環境において物理メモリ中のデータからVMのメモリ空間を復元する手法を提案し,有効性を確認した.
    \item その結果,従来は困難であったDMAによるステルス性の高いマルウェア検知,解析手法をKVM仮想マシンに適用することが可能となった.

\end{enumerate}


\section{本稿の構成}

本稿の構成は以下の通りである.第\ref{chap:related_works}章では関連研究・技術について
概説した後に既存のメモリフォレンジックをVMに適用する手法や,その問題点について
述べる.第\ref{chap:approach}章ではKVM上のVMにメモリフォレンジックを適用する
ための提案手法の概観について述べ,第\ref{chap:implementation}章ではその詳細な
実装について解説する.第\ref{chap:evaluation}章で提案手法の有効性を評価し,
第\ref{chap:conclusion}章にて実験結果についての議論,今後の展望を含む
本研究の総括を行う.

